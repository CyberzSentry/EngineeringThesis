% !TeX spellcheck = en_GB
%%%%%%%%%%%%%%%%%%%%%%%%%%%%%%%%%%%%%%%%%%
%                                        %
%    Engineer thesis LaTeX template      % 
%                                        %
%%%%%%%%%%%%%%%%%%%%%%%%%%%%%%%%%%%%%%%%%%



\documentclass[a4paper,twoside,12pt]{book}
\usepackage[utf8]{inputenc}                                      
\usepackage[T1]{fontenc}  
\usepackage{amsmath,amsfonts,amssymb,amsthm}
\usepackage[polish,british]{babel} 
\usepackage{indentfirst}
\usepackage{lmodern}
\usepackage{graphicx} 
\usepackage{hyperref}
\usepackage{booktabs}
\usepackage{tikz}
\usepackage{pgfplots}
\usepackage{mathtools}
\usepackage{geometry}
\usepackage[page]{appendix} 

\usepackage{setspace}
\onehalfspacing


\frenchspacing

\usepackage{listings}
\lstset{
	language={},
	basicstyle=\ttfamily,
	keywordstyle=\lst@ifdisplaystyle\color{blue}\fi,
	commentstyle=\color{gray}
}

%%%%%%%%%

 

%%%%%%%%%%%% FANCY HEADERS %%%%%%%%%%%%%%%

\usepackage{fancyhdr}
\pagestyle{fancy}
\fancyhf{}
\fancyhead[LO]{\nouppercase{\it\rightmark}}
\fancyhead[RE]{\nouppercase{\it\leftmark}}
\fancyhead[LE,RO]{\it\thepage}


\fancypagestyle{onlyPageNumbers}{%
   \fancyhf{} 
   \fancyhead[LE,RO]{\it\thepage}
}

\fancypagestyle{PageNumbersChapterTitles}{%
   \fancyhf{} 
   \fancyhead[LO]{\nouppercase{\it\rightmark}}
   \fancyhead[RE]{\nouppercase{\it\leftmark}}
   \fancyhead[LE,RO]{\it\thepage}
}


%%%%%%%%%%%%%%%%%%%%%%%%%%%
% listings 
\usepackage{listings}
\lstset{%
language=C++,%
commentstyle=\textit,%
identifierstyle=\textsf,%
keywordstyle=\sffamily\bfseries, %\texttt, %
%captionpos=b,%
tabsize=3,%
frame=lines,%
numbers=left,%
numberstyle=\tiny,%
numbersep=5pt,%
breaklines=true,%
morekeywords={descriptor_gaussian,descriptor,partition,fcm_possibilistic,dataset,my_exception,exception,std,vector},%
escapeinside={@*}{*@},%
%texcl=true, % wylacza tryb verbatim w komentarzach jednolinijkowych
}
%%%%%%%%%%%%%%%%%%%%%%%%%%%%%%%%%%%%

%%%% TODO LIST GENERATOR %%%%%%%%%

\usepackage{color}
\definecolor{brickred}      {cmyk}{0   , 0.89, 0.94, 0.28}

\makeatletter \newcommand \kslistofremarks{\section*{Remarks} \@starttoc{rks}}
  \newcommand\l@uwagas[2]
    {\par\noindent \textbf{#2:} %\parbox{10cm}
{#1}\par} \makeatother


\newcommand{\remark}[1]{%
{%\marginpar{\textdbend}
{\color{brickred}{[#1]}}}%
\addcontentsline{rks}{uwagas}{\protect{#1}}%
}

%%%%%%%%%%%%%% END OF TODO LIST GENERATOR %%%%%%%%%%% 

% some issues...

\newcounter{PagesWithoutNumbers}

\newcommand{\hcancel}[1]{%
    \tikz[baseline=(tocancel.base)]{
        \node[inner sep=0pt,outer sep=0pt] (tocancel) {#1};
        \draw[red] (tocancel.south west) -- (tocancel.north east);
    }%
}%

\newcommand{\MonthName}{%
  \ifcase\the\month
  \or January% 1
  \or February% 2
  \or March% 3
  \or April% 4
  \or May% 5
  \or June% 6
  \or July% 7
  \or August% 8
  \or September% 9
  \or October% 10
  \or November% 11
  \or December% 12
  \fi}


%%%%%%%%%%%%%%%%%%%%%%%%%%%%%%%%%%%%%%%%%%%%%%
% Helvetica font macros for the title page:
\newcommand{\headerfont}{\fontfamily{phv}\fontsize{18}{18}\bfseries\scshape\selectfont}
\newcommand{\titlefont}{\fontfamily{phv}\fontsize{18}{18}\selectfont}
\newcommand{\otherfont}{\fontfamily{phv}\fontsize{14}{14}\selectfont}

%%%%%%%%%%%%%%%%%%%%%%%%%%%%%%%%%%%%%%%%%%%%%%
%%%%%%%%%%%%%%%%%%%%%%%%%%%%%%%%%%%%%%%%%%%%%%
%%%%%%%%%%%%%%%%%%%%%%%%%%%%%%%%%%%%%%%%%%%%%%
%%%%%%%%%%%%%%%%%%%%%%%%%%%%%%%%%%%%%%%%%%%%%%
%%%%%%%%%%%%%%%%%%%%%%%%%%%%%%%%%%%%%%%%%%%%%%
%%%%%%%%%%%%%%%%%%%%%%%%%%%%%%%%%%%%%%%%%%%%%%
%%%%%%%%%%%%%%%%%%%%%%%%%%%%%%%%%%%%%%%%%%%%%%


\newcommand{\Author}{Maksym Brzęczek}
\newcommand{\Supervisor}{Błażej Adamczyk, DSc PhD}
\newcommand{\Consultant}{Michał Kawulok, PhD}
\newcommand{\Title}{Sensitive data extraction tool}
\newcommand{\Polsl}{Silesian University of Technology}
\newcommand{\Faculty}{Faculty of Automatic Control, Electronics and Computer Science}


\begin{document} 
	
%%%%%%%%%%%%%%%%%%  Title page %%%%%%%%%%%%%%%%%%% 
\pagestyle{empty}
{
	\newgeometry{top=2.5cm,%
	             bottom=2.5cm,%
	             left=3cm,
	             right=2.5cm}
	\sffamily
	\rule{0cm}{0cm}
	
	\begin{center}
	\includegraphics[width=29mm]{polsl}
	\end{center} 
	\vspace{1cm}
	\begin{center}
	\headerfont \Polsl
	\end{center}
	\begin{center}
	\headerfont \Faculty
	\end{center}
	\vfill
	\begin{center}
	\titlefont Engineer  thesis
	\end{center}
	\vfill
	
	\begin{center}
	\otherfont \Title\par
	\end{center}
	
	\vfill
	
	\vfill
	 
	\noindent\vbox
	{
		\hbox{\otherfont author: \Author}
		\vspace{12pt}
		\hbox{\otherfont supervisor: \Supervisor}
		\vspace{12pt}
		\hbox{\otherfont consultant: \Consultant}
	}
	\vfill 
 
   \begin{center}
   \otherfont Gliwice,  \MonthName\ \the\year
   \end{center}	
	\restoregeometry
}
  

\cleardoublepage
 

\rmfamily
\normalfont


%%%%%%%%%%%% statements required by law and Dean's office %%%%%%%%%%
\cleardoublepage

\begin{flushright}
załącznik nr 2 do zarz. nr 97/08/09 
\end{flushright}

\vfill  

\begin{center}
\Large\bfseries Oświadczenie
\end{center}

\vfill

Wyrażam  zgodę / Nie wyrażam zgody*  na  udostępnienie  mojej  pracy  dyplomowej / rozprawy doktorskiej*.

\vfill

Gliwice, dnia {\selectlanguage{polish}\today}

\vfill

\rule{0.5\textwidth}{0cm}\dotfill 

\rule{0.5\textwidth}{0cm}
\begin{minipage}{0.45\textwidth}
{\begin{center}(podpis)\end{center}}
\end{minipage} 

\vfill

\rule{0.5\textwidth}{0cm}\dotfill 

\rule{0.5\textwidth}{0cm}
\begin{minipage}{0.45\textwidth}
{\begin{center}\rule{0mm}{5mm}(poświadczenie wiarygodności podpisu przez Dziekanat)\end{center}}
\end{minipage}


\vfill

* podkreślić właściwe

 


%%%%%%%%%%%%%%%%%%%%%  
\cleardoublepage

\rule{1cm}{0cm}

\vfill  

\begin{center}
\Large\bfseries Oświadczenie promotora
\end{center}

\vfill

Oświadczam, że praca „\Title” spełnia wymagania formalne pracy dyplomowej inżynierskiej.

\vfill



\vfill

Gliwice, dnia {\selectlanguage{polish}\today}

\rule{0.5\textwidth}{0cm}\dotfill 

\rule{0.5\textwidth}{0cm}
\begin{minipage}{0.45\textwidth}
{\begin{center}(podpis promotora)\end{center}}
\end{minipage} 

\vfill
 
 

\cleardoublepage


%%%%%%%%%%%%%%%%%% Table of contents %%%%%%%%%%%%%%%%%%%%%%
\pagenumbering{Roman}
\pagestyle{onlyPageNumbers}
\tableofcontents

%%%%%%%%%%%%%%%%%%%%%%%%%%%%%%%%%%%%%%%%%%%%%%%%%%%%%
\setcounter{PagesWithoutNumbers}{\value{page}}
\mainmatter
\pagestyle{PageNumbersChapterTitles}

%%%%%%%%%%%%%% body of the thesis %%%%%%%%%%%%%%%%%


\chapter{Introduction}

This chapter presents the problem that the project tries to solve and presents the document structure.

\section{Description of the problem}

The invention and propagation of the internet has boosted the ways in which technology impacts
all of us. In this age an increasing amount of our everyday life is digitized and dependent 
on cybernetic systems hosted and operated by independent corporations and institutions. Significant
amount of our tasks has become automated through information technology solutions. The phisical word 
surrounding us also becomes intertwined with technology. We are gradualy connecting things of everyday 
use like cars or house locks to the web through Internet of Things technologies. This process has 
made our lives simpler and allowed us to achieve amazing things but it has also made us vulnerable to 
cybernetic attacks. It is only natural that the rize of the impact of technology was followed 
by the rise in the cyber crime and cyber security providers \cite{bib:articleImportanceOfCybersecurity}. 

Over the years an ecosystem has emerged that constantly competes with malicious hackers to keep us all secure.
One of the elements of this structure is penetration testing also known as ethical hacking. In it's core, 
this practice is simply simulating a real attack. There are multiple sources that depict approaches 
used to perform this process. One of the common denominators between all of them is the importance of 
gathering information \cite{bib:bookEthicalHacking}. The reason for that is because the more information you can uncover and analyze, 
the bigger the chance of finding vulnerable systems or flaws in them. One of the clusters of information 
in in companies is a communication channel like slack or discord. There are manny situations where employees 
share information connected to projects and their workplace enviroment. If an attacker was to acces such a 
platform he could potentially analyse the conversation history in search of sensitive information like ip addresses, 
logins, passwords, emails, phone numbers, etc.

Such information is especially important from a legal point of view. Introduction of General Data Protection
Regulation in 2016 has put a preassure on manny legal bodies to responsibly handle people's personal data under
a threat of heavy financial penalties \cite{bib:bookRODO}. Monitoring of the data located in the private entity's internal communication
channel might prove very useful in fulfilling the legislative requirements of private information processing.

Unfortunately such a task may be very time and resource consuming. A tool capable of scanning the history of 
communication channel in search of data that would meet some established criteria could however fulfill this job or at least
increase efficiency of the person responsible for it.

\section{Project scope}

This thesis focuses on the research into proper implementation of a cyber security tool with the purpose of processing 
a large amount of natural language messages in search of data that can be classified as personal or sensitive from the cyber security
point of view. Special focus is put on possibility of calibration and customisation of the search engine and it's reusability, regardless of 
circumstances and enviroment.

The project will primarily focus on finding nine categories of information:

\begin{itemize}
   \item IP addresses - numerical label assigned to each device connected to a computer network that uses the Internet Protocol for communication \cite{bib:articleIP}.
   \item National identification numbers - numbers issued by a Government Entity as a means of providing Identification of their citizens and or Aliens\cite{bib:internetIdentityNumber}.
   \item ID card numbers - serial numbers of documents (such as a cards) bearing identifying information about and often a photograph of the individual whose 
   name appears on it\cite{bib:internetID}. 
   \item MAC addresses - unique identifier for an Ethernet or network adapter over a network. It distinguishes different network interfaces and is used for a number 
   of network technologies, particularly most IEEE 802 networks, including Ethernet \cite{bib:internetMAC}.
   \item Domain names - combinations of letters and numbers used in combination of the various domain name extensions, such as .com, .net to find and 
   identify computers on the Internet\cite{bib:internetDomain}.
   \item Email addresses - serieses of letters, numbers, and symbols used to send and receive email \cite{bib:internetEmail}.
   \item Passwords - secret words or combinations of letters or numbers, used for communicating with another person or with a computer to prove who you are \cite{bib:internetPassword}.
   \item Usernames - a unique sequence of characters used by a person with access to a computer, network, or online service \cite{bib:internetUsername}.
   \item Phone numbers - numbers assigned to a telephone line for a specific phones or set of phones (as for a residence) that are used to call those phones \cite{bib:internetPhone}.
   \item Additional - data types specified and implemented by the users.
\end{itemize}


\section{Description of chapters}

This document is divided into seven chapters with following focus:

\begin{itemize}
   \item Chapter 1 Introduction - Presentation of the problem domain and scope, introduction of the document structure.
   \item Chapter 2 Problem analysis - Research of the topic and design of the program.
   \item Chapter 3 Requirements and tools - Description of the used technologies and required prerequisites.
   \item Chapter 4 External specification - Instruction of the program usage.
   \item Chapter 5 Internal specification - Elaboration on the programs internal structure.
   \item Chapter 6 Verification and validation - Presentation of the testing methodology and results.
   \item Chapter 7 Conclusions - Final remarks and conclusions for the future of the project.
   \end{itemize}

\chapter{Problem analysis}

\section{Existing solutions}

There are not manny widely available solutions that are capable of performing the job presented in the thesis introduction. However the scope of the project shares
similarities with Data Lekage Prevention Systems which focus on analysis of the content of confidential data and the surrounding context in order to
prevent unwanted disclosures of information. Those programs usually use up to three techniques to monitore the sensitive information - regular expressions (regexes),
data fingerprinting and statistical analysis \cite{bib:articleDLPS}. This approach could be adapted for the purposes of this thesis. 

Regular expressions are an abstraction of key-word search that enables the identification of text using apattern instead of an exact string. They are a tool 
frequently used for parsing users input and capturing parts of strings \cite{bib:conferenceRegex}. Regexes are used in Data Leakage Prevention Systems for detecting 
data like credit card numbers and social security mumbers \cite{bib:articleDLPS} which belong to the scope of this thesis. It might be possible to extend this approach 
in to other categories of data covered by the designed program.

Regular expressions are however known to have high false positives rates \cite{bib:articleDLPS}. It might be advantageous for the purposes of this project 
to take under consideration additional properties of the considered data types. It is possible for the personal information issued
by some entities to implement a check sum algorithm used for validation of authenticity. Implementing a adjustable additional check 
of the data discovered by regular expressions could possibly reduce the amount of false positives. Such feature could be utilised by 
individual user to further enhance accuracy in any case where identifing additional patterns, that escape regular expression domain, is possible.

While dictionary search does not seem to be a common part of Data Leakage Prevention Systems it might prove useful in achieving the goal of this thesis.
The natural language messages processed by the developed program may contain keywords indicating presence of the desired data, which might have not been 
detected by the initial regex search. Implementing a dictionary search which focuses on words like "password", "login" etc. could potentially increase the
reliability of the solution. 

\section{Modularity and Customisation}

Each usecase of the designed system might differ depending on the user, enviroment and multitude of other conditions. 
Some clients of the program might be interested in searching for a subset of data types implemented in the program. It is
also possible that an unanticipated in the design information type might be of a particular interest for a user. 

In order to mitigate those problems a modular approach to the design of program could be taken. Its main goal would be to allow specifing
which of the stock implemented data types to search for as well as simplifying the augmentation of the program's scope.

It is also crucial to take under consideration the fact that the structure of some data types can differ significantly between uses.
An example of this is a personal identification number which may vary depending on issuing authority. Giving the user a easily accessible possibility to 
adjust the regular expressions used in searching process as well as addictional check functions may increase the use cases of the solutions.

\section{Result presentation}

While the focus of this thesis is a program which focuses on finding some sensitive data, it might be useful to put a special emphasis on the
presentation of the results. Information discovered by the solution might be riddled with false positives. In such case giving the user a possibility
to easily analyse the message and conversation context of found data could increase the usefulness of the tool. 

\section{Enviroment }

\chapter{Requirements and tools}

\section{Tools}
In order to achieve the goal of this thesis following tools were used:
\begin{itemize}
   \item Visual Studio Code - a source-code editor developed by Microsoft for Windows, 
   Linux and macOS. It's source code is free and open source and 
   released under the permissive MIT License. It was used with a Python plugin which provides a rich support for the Python language , including 
   features such as IntelliSense, linting, debugging, code navigation, code formatting, Jupyter notebook support, refactoring, variable explorer, 
   test explorer, snippets, and more \cite{bib:internetVSC}\cite{bib:internetVSCLicence}.
   \item Qt Designer - tool for designing and building graphical user interfaces (GUIs) with Qt Widgets. Allows for composing and customization 
   of windows or dialogs in a what-you-see-is-what-you-get (WYSIWYG) manner, and testing of them using different styles and resolutions\cite{bib:internetQt}.
\end{itemize} 
\section{Technologies}
Entire project is developed with use of following technologies:
\begin{itemize}
   \item Python 3.7.3 - an easy to learn, powerful programming language. It has efficient high-level data structures 
   and a simple but effective approach to object-oriented programming. Python’s elegant syntax and dynamic typing, together with its interpreted nature, make it 
   an ideal language for scripting and rapid application development in many areas on most platforms.The Python interpreter and the extensive standard library 
   are freely available in source or binary form for all major platforms from the Python Web site, https://www.python.org/, and may be freely distributed.   
   The same site also contains distributions of and pointers to many free third party Python modules, programs and tools, and additional documentation \cite{bib:bookPython}.
\end{itemize}
\section{Functional requirements}

The program designed in this thesis is supposed to:
\begin{itemize}
   \item Perform a regex search on natural language message sets.
   \item Additionally check data found with regular expressions with false positives check function if one is available.
   \item Perform dictionary search on natural language message sets.
   \item Display the found data.
   \item Allow for displaying of message context of found data.
   \item Allow for implementation of custom searches.
   \item Save to and load results from a output file.
\end{itemize}

\section{Nonfunctional requirements}


% \begin{itemize}
% \item functional and nonfunctional requirements
% \item use cases (UML diagrams)
% \item description of tools
% \item methodology of design and implementation
% \end{itemize} 


\chapter{External specification}
\begin{itemize}
\item hardware and software requirements
\item installation procedure
\item activation procedure
\item types of users
\item user manual
\item system administration
\item security issues % program is not safe!!!
\item example of usage
\item working scenarios (with screenshots or output files)
\end{itemize}

\begin{figure}
\centering
\begin{tikzpicture}
\begin{axis}[
    y tick label style={
        /pgf/number format/.cd,
            fixed,   
            fixed zerofill, % 1.0 instead of 1
            precision=1,
        /tikz/.cd
    },
    x tick label style={
        /pgf/number format/.cd,
            fixed,
            fixed zerofill,
            precision=2,
        /tikz/.cd
    }
]
\addplot [domain=0.0:0.1] {rnd};
\end{axis} 
\end{tikzpicture}
\caption{A caption of a figure is \textbf{below} it.}
\label{fig:2}
\end{figure}


\chapter{Internal specification}

\begin{itemize}
\item concept of the system
\item system architecture
\item description of data structures (and data bases)
\item components, modules, libraries, resume of important classes (if used)
\item resume of important algorithms (if used)
\item details of implementation of selected parts
\item applied design patterns
\item UML diagrams
\end{itemize}


Use special environment for inline code, eg \lstinline|descriptor| or \lstinline|descriptor_gaussian|. 
Longer parts of code put in the figure environment, eg. code in Fig. \ref{fig:pseudokod}. Very long listings–move to an appendix.

\begin{figure}
\centering
\begin{lstlisting}
class descriptor_gaussian : virtual public descriptor
{
   protected:
      /** core of the   set */
      double _mean;
      /** fuzzyfication of the gaussian fuzzy set */
      double _stddev;
      
   public:
      /** @param mean core of the set
          @param stddev standard deviation */
      descriptor_gaussian (double mean, double stddev);
      descriptor_gaussian (const descriptor_gaussian & w);
      virtual ~descriptor_gaussian();
      virtual descriptor * clone () const;
      
      /** The method elaborates membership to the gaussian fuzzy set. */
      virtual double getMembership (double x) const;
     
};
\end{lstlisting}
\caption{The \lstinline|descriptor_gaussian| class.}
\label{fig:pseudokod}
\end{figure}


\chapter{Verification and validation}
\begin{itemize}
\item testing paradigm (eg V model)
\item test cases, testing scope (full / partial)
\item detected and fixed bugs
\item results of experiments (optional)
\end{itemize}

 
 

\chapter{Conclusions}

% Modification of results after presentation
% Point system forrating results
% Low python gui capabilities


\begin{itemize}
\item achieved results with regard to objectives of the thesis and requirements
\item path of further development (eg functional extension …)
\item encountered difficulties and problems
\end{itemize}

 
\begin{table}
\centering
\caption{A caption of a table is \textbf{above} it.}
\label{id:tab:wyniki}
\begin{tabular}{rrrrrrrr}
\toprule
	         &                                     \multicolumn{7}{c}{method}                                      \\
	         \cmidrule{2-8}
	         &         &         &        \multicolumn{3}{c}{alg. 3}        & \multicolumn{2}{c}{alg. 4, $\gamma = 2$} \\
	         \cmidrule(r){4-6}\cmidrule(r){7-8}
	$\zeta$ &     alg. 1 &   alg. 2 & $\alpha= 1.5$ & $\alpha= 2$ & $\alpha= 3$ &   $\beta = 0.1$  &   $\beta = -0.1$ \\
\midrule
	       0 &  8.3250 & 1.45305 &       7.5791 &    14.8517 &    20.0028 & 1.16396 &                       1.1365 \\
	       5 &  0.6111 & 2.27126 &       6.9952 &    13.8560 &    18.6064 & 1.18659 &                       1.1630 \\
	      10 & 11.6126 & 2.69218 &       6.2520 &    12.5202 &    16.8278 & 1.23180 &                       1.2045 \\
	      15 &  0.5665 & 2.95046 &       5.7753 &    11.4588 &    15.4837 & 1.25131 &                       1.2614 \\
	      20 & 15.8728 & 3.07225 &       5.3071 &    10.3935 &    13.8738 & 1.25307 &                       1.2217 \\
	      25 &  0.9791 & 3.19034 &       5.4575 &     9.9533 &    13.0721 & 1.27104 &                       1.2640 \\
	      30 &  2.0228 & 3.27474 &       5.7461 &     9.7164 &    12.2637 & 1.33404 &                       1.3209 \\
	      35 & 13.4210 & 3.36086 &       6.6735 &    10.0442 &    12.0270 & 1.35385 &                       1.3059 \\
	      40 & 13.2226 & 3.36420 &       7.7248 &    10.4495 &    12.0379 & 1.34919 &                       1.2768 \\
	      45 & 12.8445 & 3.47436 &       8.5539 &    10.8552 &    12.2773 & 1.42303 &                       1.4362 \\
	      50 & 12.9245 & 3.58228 &       9.2702 &    11.2183 &    12.3990 & 1.40922 &                       1.3724 \\
\bottomrule
\end{tabular}
\end{table}  

 

 


%%%%%%%%%%%%%%%%%%%%%%%%%%%%%%%%%%%%%%%%%%
\backmatter
\pagenumbering{Roman}
\stepcounter{PagesWithoutNumbers}
\setcounter{page}{\value{PagesWithoutNumbers}}

\pagestyle{onlyPageNumbers}

%%%%%%%%%%% bibliography %%%%%%%%%%%%
\bibliographystyle{plain}
\bibliography{bibliography}

%%%%%%%%%  appendices %%%%%%%%%%%%%%%%%%% 

\begin{appendices} 


 

\chapter*{List of abbreviations and symbols}

\begin{itemize}
\item[DNA] deoxyribonucleic acid
\item[MVC] model--view--controller 
\item[$N$] cardinality of data set
\item[$\mu$] membership function of a fuzzy set
\item[$\mathbb{E}$] set of edges of a graph
\item[$\mathcal{L}$] Laplace transformation
\end{itemize}


\chapter*{Listings}

(Put long listings in the appendix.)

\begin{lstlisting}
partition fcm_possibilistic::doPartition
                             (const dataset & ds)
{
   try
   {
      if (_nClusters < 1)
         throw std::string ("unknown number of clusters");
      if (_nIterations < 1 and _epsilon < 0)
         throw std::string ("You should set a maximal number of iteration or minimal difference -- epsilon.");
      if (_nIterations > 0 and _epsilon > 0)
         throw std::string ("Both number of iterations and minimal epsilon set -- you should set either number of iterations or minimal epsilon.");
   
      auto mX = ds.getMatrix();
      std::size_t nAttr = ds.getNumberOfAttributes();
      std::size_t nX    = ds.getNumberOfData();
      std::vector<std::vector<double>> mV;
      mU = std::vector<std::vector<double>> (_nClusters);
      for (auto & u : mU)
         u = std::vector<double> (nX);
      randomise(mU);
      normaliseByColumns(mU);
      calculateEtas(_nClusters, nX, ds);
      if (_nIterations > 0)
      {
         for (int iter = 0; iter < _nIterations; iter++)
         {
            mV = calculateClusterCentres(mU, mX);
            mU = modifyPartitionMatrix (mV, mX);
         }
      }
      else if (_epsilon > 0)
      {
         double frob;
         do 
         {
            mV = calculateClusterCentres(mU, mX);
            auto mUnew = modifyPartitionMatrix (mV, mX);
            
            frob = Frobenius_norm_of_difference (mU, mUnew);
            mU = mUnew;
         } while (frob > _epsilon);
      }
      mV = calculateClusterCentres(mU, mX);
      std::vector<std::vector<double>> mS = calculateClusterFuzzification(mU, mV, mX);
      
      partition part;
      for (int c = 0; c < _nClusters; c++)
      {
         cluster cl; 
         for (std::size_t a = 0; a < nAttr; a++)
         {
            descriptor_gaussian d (mV[c][a], mS[c][a]);
            cl.addDescriptor(d);
         }
         part.addCluster(cl);
      }
      return part;
   }
   catch (my_exception & ex)                                  
   {                                                       
      throw my_exception (__FILE__, __FUNCTION__, __LINE__, ex.what()); 
   }                                                          
   catch (std::exception & ex)                                 
   {                                                            
      throw my_exceptionn (__FILE__, __FUNCTION__, __LINE__, ex.what()); 
   }                                                            
   catch (std::string & ex)                                     
   {                                                            
      throw  (__FILE__, __FUNCTION__, __LINE__, ex);        
   }                                                             
   catch (...)                                                   
   {                                                             
      throw my_exception (__FILE__, __FUNCTION__, __LINE__, "unknown expection");       
   }  
}
\end{lstlisting} 

\chapter*{Contents of attached CD}

The thesis is accompanied by a CD containing:
\begin{itemize}
\item thesis (\LaTeX\ source files and final \texttt{pdf} file),
\item source code of the application,
\item test data.
\end{itemize}
 

\listoffigures
\listoftables
	
\end{appendices}


\end{document}


%% Finis coronat opus.